\documentclass[]{article}

\usepackage{graphicx}
\usepackage{caption}
\usepackage{subfig}

%opening
\title{Systems Biology Project 2}
\author{Mathias D. Kallick}

\begin{document}

\maketitle

\section{Activator Cascade}
\begin{enumerate}
	\item Write down the ODEs that describe the system:
	\begin{center}
		\begin{tabular}{ l c l}
			$\frac{dY}{dt}$ & $=$ & $V_{Ymax} (\frac{X^{*nY}}{K_Y^{nY} + X^{*nY}}) - Y dy$ \\
			$\frac{dZ}{dt}$ & $=$ & $V_{Zmax} (\frac{Y^{nZ}}{K_Z^{nZ} + Y^{nZ}}) - Z dz$ \\
		\end{tabular}
	\end{center}
	\item Plot of Activator Cascade: See Figure \ref{fig:activator cascade}
		\begin{figure}[!htbp]
			\includegraphics[width=\linewidth]{{"../plots/Activator Cascade"}.png}
			\caption{Plot of two molecules with an activator $X^*$. Parameters: $vYmax = 1 \frac{nM}{s}$, $KY = 0.5 nM$, $nY = 2$, $vZmax = 1nM/s, KZ = 0.5 \frac{nM}{s}$, $nZ = 2$, and $d = 0.1$.}
			\label{fig:activator cascade}
		\end{figure}
	\item  Explore the effects of signal strength (i.e. the magnitude of X*) on the profile of Z. When does a stronger signal fail to yield a higher response? Why? \\ \\
	When we increase the magnitude of X*, the profile of Z does not change, but the ratio of the steady state of Y to Z increases. That is, the steady state of Y increases until it is closer to that of Z. The biggest change happens from $X* = 1$ to $X* = 2$. This is because the value of $X*$ is squared when it is used, which has no effect on $1$ but does affect $2$. Further, that first step is the biggest ratio step (times $4$). See Figure \ref{fig:activator cascade grid} for graphs with values of $X*$ from $1$ to $9$.
	\begin{figure}[!htbp]
		\centering
		\subfloat{
			\includegraphics[width=40mm]{{"../plots/Activator Cascade (X=1)"}.png}
		}
		\subfloat{
			\includegraphics[width=40mm]{{"../plots/Activator Cascade (X=2)"}.png}
		}
		\subfloat{
			\includegraphics[width=40mm]{{"../plots/Activator Cascade (X=3)"}.png}
		}
		\hspace{0mm}
		\subfloat{
			\includegraphics[width=40mm]{{"../plots/Activator Cascade (X=4)"}.png}
		}
		\subfloat{
			\includegraphics[width=40mm]{{"../plots/Activator Cascade (X=5)"}.png}
		}
		\subfloat{
			\includegraphics[width=40mm]{{"../plots/Activator Cascade (X=6)"}.png}
		}
		\hspace{0mm}
		\subfloat[]{   % ???
			\includegraphics[width=40mm]{{"../plots/Activator Cascade (X=7)"}.png}
		}
		\subfloat{
			\includegraphics[width=40mm]{{"../plots/Activator Cascade (X=8)"}.png}
		}
		\subfloat{
			\includegraphics[width=40mm]{{"../plots/Activator Cascade (X=9)"}.png}
		}
		\caption{Plots of two molecules with an activator $X^*$. Parameters: $vYmax = 1 \frac{nM}{s}$, $KY = 0.5 nM$, $nY = 2$, $vZmax = 1nM/s, KZ = 0.5 \frac{nM}{s}$, $nZ = 2$, and $d = 0.1$. The value of activated $X^*$ was changed between $1$ and $9$.}
		\label{fig:activator cascade grid}
	\end{figure}
	\item If we increase the Hill coefficient, how does the answer to (3) change? Why? \\ \\
	Nothing really changes as we increase the Hill coefficient because we always rely on an equation that contains the protein concentration in the numerator and the denominator. \\
	\item What is the relationship between the response time of Z and the onset of the signal? Numerically determine Z’s response time by estimating its steadystate ZSS and the time it takes for Z to become ZSS/2 (in the original configuration and others if you would like to explore further). \\ \\
	The response time of Z in the original configuration is 8.01 seconds. If we increase the value of $X*$, the response time decreases (response time = 7.84 s for $X* = 2$, 7.81 for $X=3$, etc).
\end{enumerate}

\section{Feed-Forward Loop}
\begin{enumerate}
	\item Write down the ODEs that describe the system:
	\begin{center}
		\begin{tabular}{ l c l}
			$\frac{dY}{dt}$ & $=$ & $V_{Ymax} (\frac{X^{*nY}}{K_Y^{nY} + X^{*nY}}) - Y dy$ \\
			$\frac{dZ}{dt}$ & $=$ & $V_{Zmax} (\frac{X^{*nY}}{K_Y^{nY} + X^{*nY}}) (\frac{Y^{nZ}}{K_Z^{nZ} + Y^{nZ}}) - Z dz$ \\
		\end{tabular}
	\end{center}
	\item Plot of Activator Cascade:
	\begin{figure}[!htbp]
		\includegraphics[width=\linewidth]{{"../plots/Feed-Forward Loop"}.png}
		\caption{Plot of two molecules with an activator $X^*$, which also feeds forward and activates $Z$. Parameters: $vYmax = 1 \frac{nM}{s}$, $KY = 0.5 nM$, $nY = 2$, $vZmax = 1 \frac{nM}{s}$, $KZX = 0.5 nM$, $nZX = 2$, $KZY = 0.5 nM$, $nZY = 2$, and $d = 1$.}
		\label{fig:feed-forward loop}
	\end{figure}
	\item Demonstrate how this feed-forward system is immune to short signals where as the cascade of activators is susceptible to it. Is a feed-forward loop immune to short signals because it delays the signal’s off-to-on transition or on-to-off transition? Or both? \\ \\
	\begin{figure}[!htbp]
		\includegraphics[width=\linewidth]{{"../plots/Feed-Forward Loop (disturbed w-o recovery)"}.png}
		\caption{Plot of two molecules with an activator $X^*$, which also feeds forward and activates $Z$. In this case, the activator has been cut off after .5 seconds of being on. Parameters: $vYmax = 1 \frac{nM}{s}$, $KY = 0.5 nM$, $nY = 2$, $vZmax = 1 \frac{nM}{s}$, $KZX = 0.5 nM$, $nZX = 2$, $KZY = 0.5 nM$, $nZY = 2$, and $d = 1$.}
		\label{fig:feed-forward loop (disturbed w-o recovery)}
	\end{figure}
	\begin{figure}[!htbp]
		\includegraphics[width=\linewidth]{{"../plots/Feed-Forward Loop (disturbed)"}.png}
		\caption{Plot of two molecules with an activator $X^*$, which also feeds forward and activates $Z$. In this case, the activator has been cut off after .5 seconds of being on, and then put on again after another .5 seconds. Parameters: $vYmax = 1 \frac{nM}{s}$, $KY = 0.5 nM$, $nY = 2$, $vZmax = 1 \frac{nM}{s}$, $KZX = 0.5 nM$, $nZX = 2$, $KZY = 0.5 nM$, $nZY = 2$, and $d = 1$.}
		\label{fig:feed-forward loop (disturbed)}
	\end{figure}
	As we can see from Figure \ref{fig:feed-forward loop (disturbed w-o recovery)}, when we cut the signal from $X*$, the resulting decrease in concentrations is very slight, with only a small downwards slope. Meanwhile, as we can see in Figure \ref{fig:feed-forward loop (disturbed)}, if we reintroduce the signal after it has been cut, the signal resumes increasing at a decent rate. Yes, $Z$ is slightly behind $Y$ in it's increase, but both are still increasing from the start.

\end{enumerate}

\section{PER Oscillation}
\begin{enumerate}
	\item Reproduce Figure 2 from "A model for circadian oscillations in the \textit{Drosophila} period protein (PER)": See Figure \ref{fig:per oscillation}
	\begin{figure}[!htbp]
		\includegraphics[width=\linewidth]{{"../plots/PER Oscillations"}.png}
		\caption{Reproduction of Figure 2 from "A model for circadian oscillations in the \textit{Drosophila} period protein (PER)", by Albert Goldbeter.}
		\label{fig:per oscillation}
	\end{figure}
	\item Varying the PER degradation rate: See Figure \ref{fig:oscillation grid vd}. As we can see, the maximal level of PER is proportional to the value of PER degradation, which is counter-intuitive, because one would expect a higher amount of degradation to correspond to less PER.
	\begin{figure}[!htbp]
		\centering
		\subfloat{
			\includegraphics[width=65mm]{{"../plots/PER Oscillations (vd=.8)"}.png}
		}
		\subfloat{
			\includegraphics[width=65mm]{{"../plots/PER Oscillations (vd=.85)"}.png}
		}
		\hspace{0mm}
		\subfloat{
			\includegraphics[width=65mm]{{"../plots/PER Oscillations (vd=.9)"}.png}
		}
		\subfloat{
			\includegraphics[width=65mm]{{"../plots/PER Oscillations (vd=1.0)"}.png}
		}
		\hspace{0mm}
		\subfloat[]{   % ???
			\includegraphics[width=65mm]{{"../plots/PER Oscillations (vd=1.05)"}.png}
		}
		\subfloat{
			\includegraphics[width=65mm]{{"../plots/PER Oscillations (vd=1.1)"}.png}
		}
		\caption{Plots with standard y-axis showing effect of changing $v_d$ (the PER degradation rate) on PER oscillations. As we can see, lower values of $v_d$ correspond to lower maximum levels of total PER ($PT$), while higher values of $v_d$ correspond to higher maximum levels of total PER. As Goldbeter points out, this is counterintuitive because $v_d$ is the degradation rate of PER, and yet increasing it also increases the maximum value of $PT$.}
		\label{fig:oscillation grid vd}
	\end{figure}
	\item Varying the mRNA degradation rate: See Figure \ref{fig:oscillation grid vm}. As we can see, the maximal level of mRNA is inversely proportional to the value of mRNA degradation, which should be intuitive to the reader.
	\begin{figure}[!htbp]
		\centering
		\subfloat{
			\includegraphics[width=65mm]{{"../plots/PER Oscillations (vm=.5)"}.png}
		}
		\subfloat{
			\includegraphics[width=65mm]{{"../plots/PER Oscillations (vm=.55)"}.png}
		}
		\hspace{0mm}
		\subfloat{
			\includegraphics[width=65mm]{{"../plots/PER Oscillations (vm=.6)"}.png}
		}
		\subfloat{
			\includegraphics[width=65mm]{{"../plots/PER Oscillations (vm=.7)"}.png}
		}
		\hspace{0mm}
		\subfloat[]{   % ???
			\includegraphics[width=65mm]{{"../plots/PER Oscillations (vm=.75)"}.png}
		}
		\subfloat{
			\includegraphics[width=65mm]{{"../plots/PER Oscillations (vm=.8)"}.png}
		}
		\caption{Plots with standard y-axis showing effect of changing $v_m$ (the mRNA degradation rate) on mRNA oscillations. As we can see, lower values of $v_m$ correspond to higher maximum levels of total mRNA ($M$), while higher values of $v_m$ correspond to lower maximum levels of total PER. This does not follow the same pattern as varying the PER degradation rate - in fact, it is more intuitive, since higher degradation rate leads to lower maximum mRNA levels.}
		\label{fig:oscillation grid vm}
	\end{figure}
\end{enumerate}

\end{document}
